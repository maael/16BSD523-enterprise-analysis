\section{Company Resources}

When founding Spotify, Daniel Ek had already gained first-hand experience from many other companies that he had worked at and even founded in the past: most notably Advertigo, an online advertising company \parencite{Alexander2013}. Ek also founded a web services company when he was fourteen, taking on many roles such as programmer, system administrator, designer and product manager, learning many skills that he would later draw upon to found Spotify \parencite{wired_influencer}.

\subsection{In the Beginning}

Spotify was first created only a decade ago in 2006 \parencite{spotify_story_so_far, spotify_whois} and originally exclusively in Sweden. In fact 2006 was actually only its initial launch, with the service opening and beginning to accept members on an invitation only basis 2 years later, in 2008 \parencite{spotify_story_so_far}.

\subsubsection{Knowledge and information}

Spotify's main information is the licensing of the music that is core to its service, allowing them to stream it to users for free or as part of a subscription model. When the company first started out the idea of giving away music for free was entirely new, and many were sceptical about the model. However in Sweden where Spotify was originally founded, the music industry was not... \todo{TODO}

\subsubsection{Key people} \label{resources:beginning:keypeople}

Spotify was co-founded by Daniel Ek and Martin Lorentzon, as the solution to their problem of driving music evolution and providing a universal jukebox, in 2006 \parencite{ft_spotify}. This is an important detail, that Spotify arose due to a love of music, and the desire of its co-founders to fill a gap that they identified themselves then resolved to fill, as it explains their drive and dedication to their product.
\par
Prior to co-founding Spotify, Daniel Ek was briefly the CEO of $\mu$torrent, and it was there that he met Ludvig Strigeus \parencite{ft_lunch_ek}. This meeting shaped how Spotify would be made at a fundamental level, as Strigeus's knowledge of peer-to-peer (P2P) file sharing networks was used in the initial design when he was first developing the application. In fact this architecture persisted as a method to deliver music to users in Spotify from its creation until 2014, as it allowed the service to scale with the demand from its growing user base \parencite{how_spotify_works, spotify_shut_down_p2p}. This P2P approach was likely phased out due to falling bandwidth prices, so that Spotify can take on the burden of increasing its own bandwidth usage instead of leveraging its end-users bandwidth to help serve up music, as well as the increase in popularity of Spotify's mobile apps \parencite{spotify_shut_down_p2p}.
\par
Another key individual hired early on in Spotify's creation was Jonathan Forster. Prior to joining Spotify Forster had experience in sales and management roles at ValueClick \parencite{forster_linkedin}, and he went on to use this experience in other similar roles that he held over his years working for Spotify. One of his original roles at Spotify was to talk to record companies and help acquire licenses and master rights, which were vital as they were needed for the music core to Spotify's services, as the music selection available directly affects the services value to their users \parencite{jforsterinterview1, jforsterinterview2}.

\subsubsection{Physical Resources}

During development, Spotify ran out of a small amount of servers in Ek's apartment \parencite{JordanCrook2015}. By 2008 at Spotify's launch, this had grown to 20 servers \parencite{Garcia2013}. As the company's userbase grew, one tough challenge was to ensure that their servers grew in paraellel, helped by P2P networking, to provide maximal uptime of music streaming services.

\subsubsection{Funding}

In the early days, Ek and Lorentzon worked from Ek's apartment, \parencite{JordanCrook2015}. The relatively small amounts of resources they required, such as servers and infrastructure it is assumed was easy affordable thanks to both of their backgrounds. The first official investment in Spotify was in October 2008. The first round of funding, Series A, received \$21.64M from four investors Creandum, Horizons Ventures, Li Ka-shing and Northzone \parencite{Broeders2016}. 

\subsection{Present Day}

Spotify has grown to be one of the largest and most well known tech companies in the world \parencite{Nusca2016}, with over 30 million paid subscribers \parencite{30m_spotify}, and over 100 million users registered worldwide \parencite{Glenday2016}.

\subsubsection{Knowledge and information}

\subsubsection{Key people}

Through all of the highs and lows, Ek has remained at the core of the company, in his role as CEO. Additionally he also became Chairman in October 2015, when Martin Lorentzon stepped down into the vice-chairman position \parencite{lunden2016}.

Jonathan Forster did however leave the company recently, in September of this year, supposedly to work on developing various start-ups that he has invested in \parencite{Glenday2016}. 

\subsubsection{Physical Resources}

Spotify now has offices in 20 locations across the globe \parencite{Spotify2016}.

Spotify is currently running off of over 5000 servers, with 140Gbps of internet capacity, from four datacentres around the world: Stockhold, London, Ashburn (US east coast), San Jose (US west coast) \parencite{Garcia2013}. Spotify owns the hardware however space is rented.

To help scale their servers, Spotify developed several tools that they have since open-sourced. Most notably this includes `helios', a Docker container orchestration framework \parencite{Github2016}, which they used to drastically increase their production efficiency \parencite{Vanlan2015}.

In early 2016, Spotify announced that they would be moving their entire platform to Google's Cloud Platform \parencite{Henderson2016}. Moving to cloud services is a growing trend for tech companies, with Netflix also transitioning to AWS in January 2016 \parencite{Brodkin2016}. Cloud services offer advantages to companies such as scale, automated deployment, cheaper costs and redundancy. \todo{This may not make sense if we remove sections} Additionally they may have improved environmental impact, as mentioned in Section \ref{environment:environmental} although this is dependent on the cloud service provider.

\subsubsection{Funding}
At the start of 2016 30 million of Spotify's 100 millions users were paying subscribers, however that number has since grown over the course of the year to be 40 million by the middle of September the same year \parencite{30m_spotify, 40m_spotify}. Despite this, Spotify continues to lose money, and have done every year since launch, never turning over a profit. In 2015 Spotify's profit was \EUR{-173M}, despite revenue of nearly \EUR{2B} \parencite{Titcomb2016}.

In March 2016, Spotify raised \$1B in debt financing to help take on rivals Apple Music, as well as Tidal and Google Music \parencite{Reynolds2016}. Conditions of this include an increasing discount on the price of public shares of the company, fuelling growing rumours of the company's likely initial public offering (IPO).

Spotify was valued at \$8.58B in 2015 \parencite{Davidson2015}.