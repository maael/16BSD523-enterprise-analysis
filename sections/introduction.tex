\section{Introduction}
Daniel Ek, and his co-founder Martin Lorentzon, created Spotify as a universal jukebox to evolve and free music. Their idea took off and Spotify now has over 100 million users world wide \parencite{100m_spotify}. Their new take on music and ownership, by changing it to a streaming, on demand, service with free and subscription based access models, was innovative \parencite{wired_influencer}, spawning other similar services in its wake. However it also caused its own set of controversies with people unable to decide if it would save or destroy the music industry \parencite{guardian_swift, guardian_save_or_destroy}. Despite this it is undeniable that Daniel Ek and Spotify have been hugely successful, bringing change to the industry.

\section{Entrepreneur Characteristics}

\subsection{The Economic Approach to Entrepreneurship}
The economic view of entrepreneurship relates to to the notion of entrepreneurs being the cause of economic development. Schumpeter believed that an entrepreneur is a `special' person \parencite[][9]{deakins2012} where they have unique skills and abilities to bring about change in the market with the introduction of new products or processes. The invention of Spotify defined a new method of legally streaming music which subsequently changed the way music is consumed today. Ek went further and effectively formed an ecosystem where users are able to share and swap music using social media \parencite{dyer2013}. Ek negotiated hard with record labels to license their music for the platform \parencite{lynskey2013} and this shows that Ek in some ways confirms to Schumpeter's definition of an entrepreneur.
\par
On the other hand, Kirzner had a different view on what makes an entrepreneur. Kirzner believes that an entrepreneur is someone  ``who is \emph{alert} to profitable opportunities for exchange" \parencite[][8]{deakins2012}. Kirzner believed that an entrepreneur is someone who has the skills and knowledge to implement the idea, effectively acting as the `middleman' between the inventor and the market \parencite[][8]{deakins2012}. Ek was able to identify a unique market space where music can be distributed affordably to consumers thus reducing piracy whilst forming a method of monetising the service so that record labels can earn royalties off the platform \parencite{bertoni2012} thus making Ek's service the middleman between the consumer and the record companies.
\par
Nowadays these definitions are not necessarily compatible to the entrepreneurs of today. Kirzner believed that entrepreneurs are people who are looking for new trade opportunities whereas in Ek's case, he has produced a platform for this exchange to be performed between consumers and music producers. Additionally, Ek produced both the idea and found the opportunity for trade which means that there is no strong weighting to one definition of an entrepreneur but rather a balance of both. 

\subsection{The Psychological Approach to Entrepreneurship}
The psychological approach of entrepreneurship relates to the notion that successful entrepreneurs possess certain characteristics in their personality. \textcite[][13]{deakins2012} explains that entrepreneurs have ``...a set of characteristics that marks them out as special..." which goes back to Schumpeters view that entrepreneurs are `special' people \parencite[][9]{deakins2012}.
\par
\textcite[][14]{deakins2012} identified a set of characteristics that many entrepreneurs possess from various sources of literature:
\begin{itemize}
    \item McCelland's need for achievement
    \item Calculated risk taker
    \item High internal locus of control
    \item Creativity
    \item Need for autonomy
    \item Ambiguity tolerance
    \item Vision
    \item Self-efficacy
\end{itemize}
The need for achievement is evident in the initial stages of forming Spotify where had to persevere with the music companies both in Europe and in the USA to the point that he literally slept outside their offices just to keep attempting to convince record producers that the music screaming concept is a good idea \parencite{lynskey2013}. Ek showed signs of being a calculated risk taker as he was introducing Spotify at a time where people were apathetic in paying for music \parencite{lynskey2013} so Ek was relying on people paying for the convenience rather than the content itself \parencite{lynskey2013}. When Ek's job application to Google was rejected \parencite{edgecliffejohnson2016} he attempted to compete with them and whilst this effort failed \parencite{edgecliffejohnson2016} it showed that he has a high internal locus of control as he felt that he could compete with Google to create a better product. Ek's invention of the Spotify service showed that he had the creativity to form a whole new way for people to listen to music legally for free \parencite{dyer2013} and showed that he is innovative by forming Spotify into a music ecosystem where people could share music with each other even in different countries without having to worry about licensing issues \parencite{dyer2013}. Ek has been clear about his vision for Spotify outlining that he didn't just want this to be a music player but rather a platform where people can listen to music and share it with others legally across borders \parencite{dyer2013}.
\par
Whilst the psychological approach aims to demonstrate the certain traits that successful entrepreneurs possess there are downfalls to this approach. Firstly there are no pre-defined traits that can make an entrepreneur successful. Not all entrepreneurs possess exactly the same traits as others such as in the case of Ek where he hasn't really demonstrated his need for autonomy, tolerance to ambiguity and self-efficacy yet today he is very successful. \textcite[][15]{deakins2012} outlined that this approach doesn't account for external environmental factors such as the focus on the Swedish government to build a digital society \parencite{lynskey2013} as without this Spotify may have never came to fruition. Whilst the psychological approach is good for finding certain factors to an entrepreneur's success, it doesn't explain the full story behind their motives. 

\subsection{The Socio-Behavioural Approach to Entrepreneurship}
The socio-behavorial approach to entrepreneurship relates to how entrepreneurs are affected my the environment and culture where they are situated. \textcite[][17]{deakins2012} state that ``...a society's culture is a more powerful influence on the extent that individuals can successfully pursue entrepreneurship.". This means that a variety of factors in society can make or break an entrepreneur. 
\par
For Ek, the business environment was quite favourable to him so much so that without this Spotify may have never existed. In the late 90s, the Swedish government decided to treat broadband as an essential utility and had schemes avaialbe so that everyone could purchase a computer \parencite{lynskey2013}. Whilst this lead to many people having access to a computer it also lead to an increase of music piracy where people could download any music they wanted illegally which lead to a generation who did not believe that they needed to pay for music \parencite{bertoni2012}. Due to the increasing rate of piracy and apathy of acquiring music legally, the music industry was not doing so well at the time Ek was developing Spotify so they were looking for something to pull themselves out of this situation and record labels in Sweden were willing to experiment \parencite{ahmed2016} which helped Ek's situation. \textcite{ahmed2016} explains that Ek's nature stemming from Swedish culture, "...calmness and a sense of continuity..." may have helped Ek in more difficult times. Essentially without the difficulties that the music industry was facing both in Sweden and around the world coupled with the push for more access to technology by the Swedish government has attributed to the success of Ek and Spotify.
\par
The socio-behavioural approach does help to explain the traits of an entrepreneur that are not covered by the economic and psychological approaches to entrepreneurship, mainly regarding the external environment and societal culture which can affect the effectiveness of an entrepreneur in their endeavours. However one must understand that all approaches to entrepreneurship must be used in tandem to gain a better understanding of their success such as in the case of Ek where all factors played a part in the success of Spotify.