\section{Introduction}
Daniel Ek, and his co-founder Martin Lorentzon, created Spotify as a universal jukebox to evolve and free music. Their idea took off and Spotify now has over 100M users worldwide \parencite{100m_spotify}. Their new take on music and ownership, changing it to a streaming, on demand, service with free and subscription access models, was innovative \parencite{wired_influencer}, spawning other similar services in its wake. However it also caused its own set of controversies with people unable to decide if it would save or destroy the music industry \parencite{guardian_swift, guardian_save_or_destroy}. Despite this it is undeniable that Ek and Spotify have been hugely successful, bringing innovation to the industry.

\section{Entrepreneur Characteristics}

\subsection{The Economic Approach to Entrepreneurship}
The economic view of entrepreneurship relates to the notion of entrepreneurs being the cause of economic development. Schumpeter believed that an entrepreneur is a `special' person \parencite[][9]{deakins2012} who has unique skills and abilities to bring about change in markets with the introduction of new products or processes. Spotify defined a new method of legally streaming music, changing the way music is consumed today, creating an ecosystem of sharing music via social media \parencite{dyer2013}. Ek negotiated hard with record labels to license their music for the platform \parencite{guardian_save_or_destroy}, showing that in some ways he confirms Schumpeter's definition of an entrepreneur.
\par
Kirzner had a different view on what makes an entrepreneur. He believes that an entrepreneur is ``\emph{alert} to profitable opportunities for exchange" \parencite[][8]{deakins2012}, as well as someone who has the skills and knowledge to implement the idea, acting as the `middleman' between the inventor and the market \parencite[][8]{deakins2012}. Ek identified a unique market space for affordable music distribution to consumers, reducing piracy whilst forming a method of monetising the service so that record labels can earn royalties \parencite{bertoni2012} thus making Ek's service the middleman between the consumer and the record companies.
\par
These definitions are not necessarily compatible to today's entrepreneurs. Kirzner believed that entrepreneurs are people looking for new trade opportunities whereas Ek produced a platform for an exchange between consumers and producers. Ek produced both the idea and identified the opportunity, demonstrating a balance of both definitions. 

\subsection{The Psychological Approach to Entrepreneurship}
The psychological approach of entrepreneurship entails the notion that successful entrepreneurs possess certain personality characteristics. \textcite[][13]{deakins2012} explains that entrepreneurs have ``...a set of characteristics that marks them out as special...", relating back to Schumpeter's view \parencite[][9]{deakins2012}.
\par
\textcite[][14]{deakins2012} identified a set of characteristics that many entrepreneurs possess from various sources of literature:
\begin{itemize}
    \item McCelland's need for achievement
    \item Calculated risk taker
    \item High internal locus of control
    \item Creativity
    \item Need for autonomy
    \item Ambiguity tolerance
    \item Vision
    \item Self-efficacy
\end{itemize}
The need for achievement is evident in the initial stages of forming Spotify where Ek had to persevere with the music companies both in Europe (later the USA) to the point that he slept outside their offices to keep attempting to convince record producers that the music streaming concept was a good idea \parencite{guardian_save_or_destroy}. This was further proven by Ek's decision to return to work after only a few months of retirement when selling Advertigo for \$1.25M at age 23 \parencite{Taube2014}. Without direction, he became depressed and had ``no idea how [he] wanted to live [his] life'' \parencite{Taube2014}.
\par
Ek showed signs of being a calculated risk taker by introducing Spotify at a time where people were apathetic in paying for music \parencite{guardian_save_or_destroy}, relying on people paying for convenience rather than the content \parencite{guardian_save_or_destroy}. When Ek's job application to Google was rejected \parencite{ft_lunch_ek} he attempted to compete with them and whilst this effort failed \parencite{ft_lunch_ek} it showed that he has a high internal locus of control, feeling that he could compete with Google. Ek's invention of Spotify showed his creativity, forming a new way for people to listen to music legally for free \parencite{dyer2013}. Ek's vision and innovation have been demonstrated by his vision for Spotify, outlining that he didn't just want this to be a music player but rather a platform where people can listen to music and share it with others legally across borders \parencite{dyer2013}.
\par
Whilst the psychological approach aims to demonstrate certain traits that successful entrepreneurs possess, there are downfalls to the approach. Firstly there are no predefined traits that can make an entrepreneur successful. Not all entrepreneurs possess exactly the same traits as others, such as Ek who has not really demonstrated his need for autonomy, tolerance to ambiguity, and self-efficacy, yet today he is very successful. \textcite[][15]{deakins2012} outlined that this approach doesn't account for external environmental factors, such as the focus on the Swedish government to build a digital society \parencite{guardian_save_or_destroy} without which Spotify may have never have come to fruition. This approach is good for finding certain factors to an entrepreneur's success, but doesn't explain the full story behind their motives.

\subsection{The Socio-Behavioural Approach to Entrepreneurship}
The socio-behavioural approach to entrepreneurship relates to how entrepreneurs are affected by the environment and culture where they are situated. \textcite[][17]{deakins2012} state that ``...a society's culture is a more powerful influence on the extent that individuals can successfully pursue entrepreneurship", meaning that a variety of factors in society can make or break an entrepreneur. 
\par
The business environment was quite favourable for Ek, so much so that without it Spotify may have never existed. In the late 90s, the Swedish government decided to treat broadband as an essential utility, making schemes available to allow everyone to purchase a computer \parencite{guardian_save_or_destroy}. This lead to many people having access to a computer, but also lead to an increase of music piracy where people could download music they wanted illegally. This lead to a generation who did not believe in paying for music \parencite{bertoni2012}. Due to the increasing rate of piracy and apathy of acquiring music legally, the music industry was not doing well at the time around Spotify's creation. They were looking for something to pull themselves out of this situation and record labels in Sweden were willing to experiment \parencite{ft_spotify}, helping Ek's situation. \textcite{ft_spotify} explains that Ek's nature stemming from Swedish culture, "...calmness and a sense of continuity..." may have helped him in difficult times. The difficulties that the music industry faced both in Sweden and around the world, coupled with the push for increased access to technology by the Swedish government, have both played a part in Spotify's success.
\par
The socio-behavioural approach helps to explain the traits of an entrepreneur not covered by the economic and psychological approaches, mainly regarding the external environment and societal culture which can impact the effectiveness of an entrepreneur in their endeavours. However it must be understood that all approaches to understanding entrepreneurship must be used in tandem to gain a better understanding, such as in the case of Ek where all factors played a part in the success of Spotify.