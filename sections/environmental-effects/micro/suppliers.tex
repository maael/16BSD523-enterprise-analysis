\subsubsection{Suppliers} \label{environment:suppliers}
As a digital service Spotify does not have any traditional suppliers, but does require two key services. The first is hosting their service and connecting it to users, via servers and bandwidth. Then there are music suppliers in the form of record labels and artists, which supply the content. Due to this Spotify can be considered fully dependent on its suppliers, which some may consider an issue \parencite{dependent_on_suppliers} however in Spotify's case they have both chosen or had no choice on the matter. In the case of the servers and bandwidth they introduced a supplier by their own decision, which will be discussed later in this section. As for music suppliers, Spotify is a service delivering content created by others, and so their very business is dependent on their music library and content.
\par 
Until recently Spotify self managed hosting its data in its own data centres, however changed recently to use Google Cloud after having considered Amazon Web Services \parencite{siliconangle_spotify_google_cloud, google_lures_spotify}. The original decision to self host was driven by the lack of any mature platforms with the desired performance and pricing \parencite{siliconangle_spotify_google_cloud}. This recently changed however with Spotify transitioning to use Google Cloud service \parencite{siliconangle_spotify_google_cloud, spotify_announce_google_cloud}. This is significant as Spotify had to realigned itself, not only locking themselves into using Google Cloud, but also having to reorganise teams of people who originally worked on their internal hosting systems as well as the code itself, to accommodate this change \parencite{forbes_why_spotify_google_cloud}. The decision behind the change was driven by Google's data offering, which Spotify plan to leverage for better music recommendations and customer service \parencite{forbes_why_spotify_google_cloud, spotify_announce_google_cloud}. Spotify appears to have a good supplier relationship with Google, having spoken highly about Google's offerings, displaying their belief by moving their infrastructure to them \parencite{spotify_announce_google_cloud}. Google can be considered to be on good terms with Spotify as they approached and convinced them to use Google Cloud despite Spotify looking into Amazon Web Services \parencite{google_lures_spotify}.
\par 
Spotify requires the music that is core to its service, and the rights to distribute it. This requires the acquisition of licenses from music labels, with the biggest 3 being Universal Music Group, Sony Music Entertainment, and Warner Music Group \parencite{spotify_lapsed_music_labels}. This licensing is the largest expense that Spotify incurs, and is increasing, with Spotify having to pay \$1.8 billion in 2015, an 85\% increase from the previous year \parencite{broken_music_industry}. It is in part this massive cost which leads to Spotify continually operating at a loss \parencite{spotify_losses}. Unfortunately there is not a fantastic supplier relationship between Spotify and music labels, as to combat its losses it is looking to push the labels revenue share down. This is problematic however as this would mean that Spotify would be paying artists less, as they receive a portion of the money paid to the labels. The amount of money that artists receive from having their music on Spotify has already caused controversy in the past and such a reduction would be unlikely to improve the situation \parencite{guardian_swift, spotify_lapsed_music_labels}.