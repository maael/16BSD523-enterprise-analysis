\par
As a digital service Spotify doesn't have any traditional suppliers, but requires two key services. The first is hosting their service and connecting to users, via servers and bandwidth. Then there are music suppliers, record labels and artists, who supply the content. Subsequently Spotify can be considered fully dependent on its suppliers, which some may consider an issue \parencite{dependent_on_suppliers} however in Spotify's case they may have no choice on the matter. In the case of the servers and bandwidth they introduced a supplier by their own decision, which will be discussed later in this section. As for music suppliers, Spotify is a service delivering content created by others so their very business is dependent on their music library and content.
\par 
Initially Spotify self-managed hosting its data in its own data centres, but recently changed to Google Cloud after having considered Amazon Web Services \parencite{siliconangle_spotify_google_cloud, google_lures_spotify}. The decision to self-host was driven by the lack of mature platforms with the desired performance and pricing \parencite{siliconangle_spotify_google_cloud}. This changed however with Spotify transitioning to use Google Cloud service \parencite{siliconangle_spotify_google_cloud, spotify_announce_google_cloud}. This is significant as Spotify not only locked themselves into using Google Cloud, but also had to reorganise internal teams to accommodate this change \parencite{forbes_why_spotify_google_cloud}. This decision was driven by Google's data offering, which Spotify plans to leverage for better music recommendations and customer service \parencite{forbes_why_spotify_google_cloud, spotify_announce_google_cloud}. Spotify and Google appear to have a good supplier relationship, having with Spotify having spoken highly about Google's offerings and changing to their services \parencite{spotify_announce_google_cloud}, and Google convincing Spotify to use Google Cloud despite Spotify looking into AWS \parencite{google_lures_spotify}.
\par 
Spotify needs both music and the rights to distribute it. This requires the acquisition of licenses from music labels, with the biggest 3 being Universal Music Group, Sony Music Entertainment, and Warner Music Group \parencite{spotify_lapsed_music_labels}. Licensing is the largest expense that Spotify incurs, and is increasing, costing them \$1.8B in 2015, an 85\% increase from the previous year \parencite{broken_music_industry}. It is in part this massive cost which leads to Spotify continually operating at a loss \parencite{spotify_losses}. Unfortunately there is a questionable supplier relationship between Spotify and music labels, as to combat its losses it is looking to push the labels revenue share down. This is problematic as Spotify would be paying artists less, as they receive a portion of the money paid to the labels. The amount of money that artists receive from having their music on Spotify has already caused controversy in the past and such a reduction would be unlikely to improve the situation \parencite{guardian_swift, spotify_lapsed_music_labels}.