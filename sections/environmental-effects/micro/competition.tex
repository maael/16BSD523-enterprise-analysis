\subsubsection{Competition}
Since its creation in 2006, a number of competitors have been spawned with a similar music streaming access based model, seen in Table \ref{table:competitors}.
\begin{table}[ht!]
\centering
 \begin{tabular}{|l|l|l|} 
 \hline
 \textbf{Service} & \textbf{Launch Date} & \textbf{Source} \\ [0.2ex] 
 \hline
 \small{Apple Music} & \small{June 30\textsuperscript{th} 2015} & \small{\parencite{apple_music_launch}} \\ 
 \hline
 \small{Amazon Music Unlimited} & \small{October 12\textsuperscript{th} 2016 (US only)} & \small{\parencite{amazon_music_launch}} \\
 \hline
 \small{Google Play Music All Access} & \small{May 15\textsuperscript{th} 2013} & \small{\parencite{google_music_all_access_launch}} \\
 \hline
 \small{Deezer} & \small{2007 (US launch in 2016)} & \small{\parencite{deezer_launch, deezer_us_launch}} \\
 \hline
 \small{Tidal} & \small{October 28\textsuperscript{th} 2014 (2015 relaunch)} & \small{\parencite{tidal_launch}} \\
 \hline
 \end{tabular}
 \caption{Spotify's Competitors}
 \label{table:competitors}
\end{table}
\par
The first three competitors have world-recognised brands backing them, helping to promote their services. Tidal is now being backed by celebrities and owned by Jay Z \parencite{tidal_launch_issues}, and aims more at those looking for high resolution music, and a fairer deal for artists by apparently paying double the royalties compared to other services \parencite{tidal_launch_issues}. Deezer does not have the backing of a brand or celebrity, however until recently it was the second largest on-demand streaming service, gaining users by often being bundled with mobile contracts \parencite{music_streaming_guardian}.
\par
There also exists threats from outside established music subscription services, such as Pandora. Pandora is an online music radio service, similar in that it connects users and music, but with a key difference where users have no control over the music besides suggesting artists or genres. Pandora however has plans to expand to allow on-demand music \parencite{pandora_subscription}, representing a significant competitor as established service in America with 78 million users, close to Spotify's own 100 million user base.