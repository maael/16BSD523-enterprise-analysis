\par
With regards to environmental factors, Spotify argues that they are environmentally friendly on the surface, however once you delve deeper this becomes questionable. Users argue that as Spotify is a non-physical service, and reduces the need for physical media such as CDs so the environmental costs of production and distribution are also avoided \parencite{spotify_green_pro}, which would imply that the company is environmentally-friendly, if only accidentally through the company's nature. 
\par
Spotify does reduce the physical medium being distributed, but there is still the matter of digitally distributing and hosting the music, none of which is free from environmental impact. They require physical networks of devices and infrastructure, such as servers and the internet, consuming power which in most cases means fossil fuels being burnt \parencite{world_energy_stats}. Fundamentally cloud computing which Spotify uses to provide its service is not so environmentally friendly, but is instead an invisible cost to the end user \parencite{internet_isnt_so_clean}. However with Spotify's move to Google's Cloud service and Google's own commitment to moving to renewable energy, Spotify's own impact on the environment is lower than expected \parencite{spotify_announce_google_cloud, google_renewable}.