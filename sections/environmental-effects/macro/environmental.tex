\subsubsection{Environmental} \label{environment:environmental}
In regards to environmental factors, Spotify has arguments that it is environmentally-friendly on the surface, however once you delve deeper in to the service this becomes questionable. Users argue that as Spotify is a non-physical service, and reduces the need for physical mediums such as CDs, it means that the environmental cost of producing and distributing such products are also avoided \parencite{spotify_green_pro}, which would imply that the company is environmentally-friendly, if only accidentally through the company's nature. 
\par
However while Spotify does reduce the physical medium being distributed, there is still the matter of digitally distributing the music and hosting the content, none of which are is free from environmental impact. They require a physical network of devices and infrastructures, such as servers and the Internet, which consume power which in most cases means oil, gas, or coal being burnt \parencite{world_energy_stats}. Fundamentally cloud computing such as that which Spotify uses to host its service and data is not so environmentally-friendly, but is instead an invisible cost to the end-user \parencite{internet_isnt_so_clean}. However what with Spotify's movement to the Google Cloud service, and Google's own commitment to moving to renewable energy, Spotify's own impact on the environment is probably substantially less than expected \parencite{spotify_announce_google_cloud, google_renewable}.